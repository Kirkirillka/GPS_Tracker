\hypertarget{performance-analysis-framework-for-base-station-placement-using-ieee-802.11}{%
\section{Performance analysis framework for base station placement using
IEEE
802.11}\label{performance-analysis-framework-for-base-station-placement-using-ieee-802.11}}

\hypertarget{introduction-and-motivation}{%
\subsection{Introduction and
Motivation}\label{introduction-and-motivation}}

The successful placement of an aerial base station requires knowledge of
the locations of the user equipment. This data must be collected
continuously to provide up-to-date information for a placement
algorithm.

To prepare the experimental evaluation of the placement of the aerial
base stations, the framework for the performance analysis needs to be
developed.

\hypertarget{requirement-overview}{%
\subsection{Requirement Overview}\label{requirement-overview}}

Its main objective is to receive the GPS location data from several
Android-based telephones that send this data via UDP sockets.

The site data collector must store the received data in an internal
database and then make this data available to the placement algorithm.
Another part of this framework is the performance evaluation module,
which sends the data over the network and records the amount of data
sent.

It is required to have a system that can receive from clients
information about their GPS coordinates and Wi-Fi signal quality. Based
on this information, an optimization algorithm must be applied, which
predicts how to change the position of the Wi-Fi station to increase
capacity concerning all connected clients.

\hypertarget{performance-analysis-framework-for-base-station-placement-using-ieee-802.11-1}{%
\section{Performance analysis framework for base station placement using
IEEE
802.11}\label{performance-analysis-framework-for-base-station-placement-using-ieee-802.11-1}}

\hypertarget{introduction-and-motivation-1}{%
\subsection{Introduction and
Motivation}\label{introduction-and-motivation-1}}

The successful placement of an aerial base station requires knowledge of
the locations of the user equipment. This data must be collected
continuously to provide up-to-date information for a placement
algorithm.

To prepare the experimental evaluation of the placement of the aerial
base stations, the framework for the performance analysis needs to be
developed.

\hypertarget{requirement-overview-1}{%
\subsection{Requirement Overview}\label{requirement-overview-1}}

Its main objective is to receive the GPS location data from several
Android-based telephones that send this data via UDP sockets.

The site data collector must store the received data in an internal
database and then make this data available to the placement algorithm.
Another part of this framework is the performance evaluation module,
which sends the data over the network and records the amount of data
sent.

It is required to have a system that can receive from clients
information about their GPS coordinates and Wi-Fi signal quality. Based
on this information, an optimization algorithm must be applied, which
predicts how to change the position of the Wi-Fi station to increase
capacity concerning all connected clients.
