\subsection{Decision on GPS\_Tracker techniques
development}\label{decision-on-gps_tracker-techniques-development}

\subsubsection{Description}\label{description}

The GPS\_Tracker is implemented in Python language. This is a dynamic
interpreting language. That is known the performance is not so perfect
compared to static compiled languages, but it has higher changeability
property.

Another problem is its dynamic nature, programmers should add more codes
to check that the variables they are working on have the right type,
right values and so one.

Also, Python doesn't have a proper parallel threading mechanism due to
GIL.

To summarize, there is a set of problems:

\begin{itemize}
\tightlist
\item
  Possible performance degradation.
\item
  Ambiguous interfaces, more safe checks.
\item
  Parallel execution, scalability is not so high.
\end{itemize}

\subsubsection{Decision}\label{decision}

In order to mitigate the consequences of using Python as the main
language for GPS\_Tracker we decided:

\begin{itemize}
\tightlist
\item
  Use additional code documentation and code annotation to clarify the
  meaning of Python code, helps IDE to derive expected behavior and
  check code validity.
\item
  Use Worker architecture pattern (\texttt{Celery} library). That will
  help to add scalability to task execution and simplify parallel code
  programming.
\item
  Use JetBrain Pycharm IDE as the main IDE for programmers to use
\end{itemize}

\subsubsection{Status}\label{status}

Accepted

\subsubsection{Consequences}\label{consequences}

\begin{itemize}
\tightlist
\item
  Worker pattern implemented in \texttt{Celery} requires additional
  component to store information about registered tasks and perform
  synchronization between workers. RabbitMQ is a message queue used to
  store that information.
\end{itemize}
