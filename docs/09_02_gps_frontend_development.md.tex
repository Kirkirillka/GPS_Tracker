\subsection{Decision on GPS\_Frontend techniques
development}\label{decision-on-gps_frontend-techniques-development}

\subsubsection{Description}\label{description}

The GPS\_Frontend is a UI to access the functions of other components of
the framework. Thus, it would have complicated behavior. The best-suited
architecture for that kind of task is Single Page Application (SPA).

There are many possible web frameworks exist that may help to build SPA
applications but the most famous and functional are:

\begin{itemize}
\tightlist
\item
  Angular
\item
  React
\item
  Vue.js
\end{itemize}

\subsubsection{Decision}\label{decision}

Only one person in the team had the experience developing a web
application with the Vue.js framework.

The team member discussed and decided that:

\begin{itemize}
\tightlist
\item
  Angular has reach functionality, but too complex for that task.
\item
  React is more flexible, but requires more time to get working in.
\item
  Vue.js is a simple framework with an excellent documentation but has a
  lack of components in the default configuration. It requires
  additional libraries and components to design required functions.
\end{itemize}

Finally, the team decided to use the Vue.js framework as the core for
\textbf{GPS\_Frontend}.

Also there is a set of dependencies specified:

\begin{itemize}
\tightlist
\item
  ``font-awesome'' - A set of fonts.
\item
  ``@fortawesome/fontawesome-free'' - A set of fonts.
\item
  ``apexcharts'' - A library do draw figures using Web.
\item
  ``axios'' - A library to work with HTTP(S) requests.
\item
  ``bootstrap'' - An open source toolkit/framework for developing with
  HTML, CSS, and JS.
\item
  ``bootstrap-vue'' - A wrapper to use Bootstrap components natively in
  Vue.js.
\item
  ``core-js'' - An standard library to extend the number of components
  and operation.
\item
  ``d3v4'' - level JavaScript library for manipulating documents based
  on data using HTML, SVG, CSS.
\item
  ``d3-colorbar'' - An extension to d3 library to work with colors.
\item
  ``jquery'' - rich JavaScript library.
\item
  ``moment'' - A convenient library to work with time.
\item
  ``pc-bootstrap4-datetimepicker'' - a DatetimePicker component
  compatible with Bootstrap.
\item
  ``plotly.js-dist'' - A library do draw figures using Web.
\item
  ``portal-vue'' - A Vue component to render your component's template
  anywhere in the DOM.
\item
  ``underscore'' - A JavaScript library that provides a whole mess of
  useful functional programming helpers without extending any built-in
  objects.
\item
  ``vue'' - The Progressive JavaScript Framework to build web
  application.
\item
  ``vue-apexcharts'' - A wrapper to use ApexChart components natively in
  Vue.js.
\item
  ``vue-axios'' - A wrapper to use Axios components natively in Vue.js.
\item
  ``vue-bootstrap-datetimepicker'' - datetimepicker components natively
  in Vue.js.
\item
  ``vue-router'' - An router extension to Vue.js.
\item
  ``vuex'' - A storage library extension to Vue.js.
\end{itemize}

\subsubsection{Status}\label{status}

Accepted

\subsubsection{Consequences}\label{consequences}

Advantages:

\begin{itemize}
\tightlist
\item
  We received a well-designed, good-looking and user-friendly UI.
\item
  It can be easily changed and adapted to new requirements.
\item
  Rich set of features and further development.
\item
  High SPA performance.
\item
  UI can be accessed easily from different devices, good portability.
\end{itemize}

Disadvantages:

\begin{itemize}
\tightlist
\item
  The difficulty of data analysis in JavaScript, that language probably
  is not the best suit for that kind of task.
\item
  Some very complex figures (like Heatmap) may busy the whole program
  drawing a large dataset.
\item
  Due to the big number of components, the JS code compilation takes a
  relatively long time.
\item
  Dependency on the user's web browser.
\item
  Vue.js is a young project, it may suffer a lack of functions provided
  by more matured frameworks.
\end{itemize}

